\documentclass[../jarvis.tex]{subfiles}
\graphicspath{{\subfix{../images/}}}

\begin{document}
\subsection{Floors and Ceilings}
In what follows, and in most modern texts, $\floor{x}$ denotes the greatest integer less than or equal to $x$. For example, $\floor{\pi}=3$ and $\floor{-1.414}=-2$. 
Similarly, $\ceiling{x}$ denotes the least integer greater than or equal to $x$. For example, $\ceiling{\pi}=4$ and $\ceiling{-1.414}=-1$.
\begin{example}[2018 SMO(S) P25]
    Suppose $R$ is a real number such that
    $$\floor{R-\frac{1}{200}}+\floor{R-\frac{2}{200}}+\cdots+\floor{R-\frac{99}{200}}=2018.$$
    Find $\floor{20R}$.
\end{example}
\begin{proof}
    By definition of the floor function,
    $$\floor{R-\frac{k}{200}}\leq R-\frac{k}{200} < \left(R-\frac{k}{200}\right)+1,$$ 
    thus, $\floor{R-\frac{1}{200}}$ and $\floor{R-\frac{99}{200}}$ differ by at most $1$ (this is the key idea!). 
    
    Define $M=\max_{1\leq k\leq 99}\floor{R-\frac{k}{200}}$, then I claim that $M\geq 20$. Suppose otherwise, then $M < 20$. Thus, 
    $$\floor{R-\frac{1}{200}}+\floor{R-\frac{2}{200}}+\cdots+\floor{R-\frac{99}{200}} < 99\cdot 20 = 1980 < 2018,$$
    a contradiction.

    Thus, $M \geq 20$. Let $a, b$ denote the number of $\floor{R-\frac{k}{200}}$ attaining the values 20 and 21 respectively such that
    $$\begin{cases}
        a+b=99 \\
        20a+21b=2018
    \end{cases}
    \implies a=61, b=38.$$

    Hence, we seek 
    $$\begin{cases}
        R-\frac{38}{200} \geq 21 \\
        R-\frac{39}{200}\leq 21
    \end{cases}
    \implies 21.19 \leq R \leq 21.195 \implies \boxed{\floor{20R}=423}$$.
\end{proof}

\subsection{A Splurge of Inequalities}
We start off this chapter with a powerful technique for proving inequalities.
\begin{example}[Schur's Inequality]\label{algebra-schur}
Let $a,b,c,r$ be positive real numbers. Prove that
$$a^r(a-b)(a-c)+b^r(b-c)(b-a)+c^r(c-a)(c-b)\geq 0.$$
\end{example}
Let $f(a,b,c)=a^r(a-b)(a-c)+b^r(b-c)(b-a)+c^r(c-a)(c-b)$. In general, we say that $f$ is symmetric if $f(a,b,c)=f(b,a,c)=f(c,b,a)=\cdots$. This means that the function remains constant even if we interchange the variables $a,b$ and $c$.

Simple enough, if we interchange $a$ and $b$, we have:
\begin{align*}
    f(b,a,c)&=b^r(b-a)(b-c)+a^r(a-c)(a-b)+c^r(c-b)(c-a) \\
    &=a^r(a-b)(a-c)+b^r(b-c)(b-a)+c^r(c-a)(c-b) \\
    &=f(a,b,c)
\end{align*}
If you are paranoid, you can manually verify this for the $3!=6$ possible "interchanges". I'll leave that to you.

With that, we say that the function, and hence Schur's Inequality, is \textbf{symmetric} in $a,b,c$. So what's the big deal?

Since the value of $f(a,b,c)$ remains constant even as we interchange variables, we may impose restrictions on $a,b,c$ that we normally cannot. In particular, we may assume \textbf{without loss of generality} that $a\geq b\geq c$ (convince yourself!). This is very useful, especially for this inequality, because we have terms in $a-b, a-c \cdots$, and the ordering of $a,b,c$ tells us whether these terms are positive or negative.

In fact, a cursory glance tells us that if $a\geq b \geq c$, then only $b^r(b-c)(b-a)$ is negative. Hence, this motivates an enlightening rearrangement of $f(a,b,c)$:
\begin{align*}
    f(a,b,c)&= a^r(a-b)(a-c)+b^r(b-c)(b-a)+c^r(c-a)(c-b)\\
            &= (a-b)[a^r(a-c)-b^r(b-c)]+c^r(a-c)(b-c) \\
\end{align*}
And we are done!
\begin{moral}
This technique, ironically, is known as "breaking symmetry". The inequality is essentially proven after we impose a specific ordering on the variables, but before that, the inequality may look almost intractible!
\end{moral}

\subsection{Selected Problems}
\problem Prove by contradiction that $\sqrt{2}$ is irrational.
\problem* Prove by contradiction that $x^2+y^2=3z^2$ has no positive integer solutions.
\problem Explain why lines \eqref{5.1-tangent-ineq} and \eqref{5.1-tangent-eq} hold in \nameref{algebra-cbc}.
(Hint: You may want to consider the density function of the normal distribution.)
\problem By referring to \nameref{algebra-schur}, prove that
\begin{itemize}
    \item $a^3+b^3+c^3\geq a^2(b+c)+b^2(c+a)+c^2(a+b),$
    \item \[\frac{1}{a^5}+\frac{1}{b^5}+\frac{1}{c^5}+\frac{a+b+c}{a^2b^2c^2}\geq \frac{b^2+c^2}{a^3b^2c^2}+\frac{c^2+a^2}{b^3c^2a^2}+\frac{a^2+b^2}{c^3a^2b^2}.\]
\end{itemize}
\problem (2017 USAJMO P2) Consider the equation $$(3x^3+xy^2)(x^2y+3y^3)=(x-y)^7.$$
\begin{itemize}
    \item Prove that there are infinitely many pairs of positive integers $(x,y)$ satisfying the equation.
    \item Describe all pairs of positive integers $(x,y)$ satisfying the equation.
\end{itemize}
\end{document}
