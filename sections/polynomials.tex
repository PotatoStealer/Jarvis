\documentclass[../jarvis.tex]{subfiles}
\graphicspath{{\subfix{../images/}}}
\begin{document}
\subsection{Polynomials \med}
Outside of quadratic polynomials, there are also cubics, quartics, quintics, sextics..., that most students assume are extinct. However, they can be found in a select number of natural hideouts!

\subsubsection{Quadratics}
In this section, we will revisit the common quadratic polynomials.
The techniques for quadratics aren't many. The usual few are the discriminant and completing the square, none of which are totally unexpected or inspiring. However, they are still useful-ish in extracting information from a given problem. We hope it suffices!
\begin{example}[2010-2011 Mandelbrot]
    Let $P(x)=x^3+ax^2+bx+c$ be a polynomial with three distinct roots. The polynomial P(Q(x)), where $Q(x)=x^2+x+2001$, has no real roots. Prove that $P(2001)>\frac{1}{64}$.
\end{example}
This is a strange idea: given the number of roots of a polynomial, what can we deduce about it's value?

\begin{proof}
    For starters, let $p,q,r$ be the three distinct roots of $P$, then $P(x)=(x-p)(x-q)(x-r)$ and let $z$ be a complex root of $P(Q(x))$ so that $P(Q(z))=0$. 

Hence, $Q(z)=\text{$p$ or $q$ or $r$}$. Without loss of generality, $$Q(z)=p \implies z^2+z+(2001-p)=0.$$

However, there are no real values of $z$ that satisfy this condition! This next step would be glaring: by considering the discriminant,
$$\Delta=1-4(2001-p) < 0 \implies p < \frac{8003}{4}.$$

Since we initially assumed without loss of generality that $Q(z)=p$, this conclusion should also hold for $q, r$, that is $p,q,r < \frac{8003}{4}$.

Finally, \begin{align*}
    P(2001)=(2001-p)(2001-q)(2001-r) &> \frac{1}{4}\cdot\frac{1}{4}\cdot\frac{1}{4} \\
    &=\frac{1}{64}
\end{align*}
\end{proof}
Next, we consider a quadratic with coefficients that form an arithmetic progression. As it turns out, if the quadratic has a unique root, then we can uniquely determine this root!
\begin{example}[2013 AMC 10B P19]
    Let $c,b,a$ be real numbers form an arithmetic progression with $a\geq b\geq c\geq 0$. The quadratic $ax^2+bx+c=0$ has exactly one root. Find this root.
\end{example}
First step! The discriminant: $\Delta=b^2-4ac=0$, implying that the only root is $x=\frac{-b}{2a}$ by considering the quadratic formula.

\begin{proof}
    Since $c,b,a$ have common differences with $b$ being the median element, we may write $c=b+(b-a)=2b-a$ so that 
$$b^2-4ac=b^2-4a(2b-a)=b^2-8ab+4a^2=0 \implies b=4a\pm2a\sqrt{3}.$$
Yet, we are given that $a\geq b$, so surely $b=4a-2a\sqrt{3}$, giving $x=-\frac{b}{2a}=\boxed{-2+\sqrt{3}}.$
\end{proof}

\subsubsection{The Rare Types}
Contrary to popular belief, there are higher degree polynomials in the wild. We first consider some identities associated with the coefficients of a general polynomial.

Consider a general polynomial $$P(x)=a_nx^n+a_{n-1}x^{n-1}+\cdots+a_1x+a_0$$ with real coefficients $\{a_n\}$. Then, the
\begin{enumerate}
    \item constant term is $P(0)$,
    \item sum of coefficients is $P(1)$,
    \item sum of \textit{odd-power} coefficients is $\frac{P(1)-P(-1)}{2}$,
    \item sum of \textit{even-power} coefficients is $\frac{P(1)+P(-1)}{2}$.
\end{enumerate}

The reader is strongly encouraged to derive the four properties above!

For the other coefficients, there are other higher-powered techniques (pun fully intended) to deal with them, but we will not introduce them here (for the interested, see the roots of unity filter).

We are guessing the student is familiar with the so-called "sum and product of roots" for a quadratic, and here, we introduce a generalisation - we have similar relations for higher-degree polynomials! This is given by Vieta's Theorem.

\begin{proposition}[Vieta's Theorem]
    A general polynomial $P(x)=a_nx^n+a_{n-1}x^{n-1}+\cdots+a_1x+a_0$ with real coefficients $\{a_n\}$ and real and complex roots $r_1, r_2, \cdots r_n$ has the following relations:
    \begin{align*}
        r_1+r_2+\cdots+r_n&=-\frac{a_{n-1}}{a_n}\\
        (r_1r_2+r_1r_3+\cdots+r_1r_n)+(r_2r_3+r_2r_4+\cdots+r_2r_n)+\cdots+r_{n-1}r_n&=\frac{a_{n-2}}{a_n} \\
        r_1r_2\cdots r_n=(-1)^n\frac{a_0}{a_n}
    \end{align*}
    In general, the \textit{sum of roots taken k at a time} is given by $(-1)^k\frac{a_{n-k}}{a_n}.$ Each of these sums are also known as \textit{elementary symmetric polynomials}.
\end{proposition}

Let us show some examples of Vieta's theorem in action. 
\begin{example}[2001 All-Russian Olympiad]
    The equation $(x-\sqrt[3]{13})(x-\sqrt[3]{53})(x-\sqrt[3]{103})=\frac{1}{3}$ has three distinct roots $r,s,t$. Find the value of $r^3+s^3+t^3$.
\end{example}

First of all... yikes! Cube roots and cubes. However, it seems that the three cube roots look arbitrary: while the answer depends on their values, we can find an expression for $r^3+s^3+t^3$ in terms of these cube roots without caring about their values just yet. 
\begin{proof}
    We make the substitution $a=\sqrt[3]{13}$, $b=\sqrt[3]{53}$, $c=\sqrt[3]{103}$, so that our equation is transformed into $$(x-a)(x-b)(x-c)-\frac{1}{3}=0$$. By Vieta's theorem, 
\begin{align*}
    r+s+t&=a+b+c \\
    rs+st+rt&=ab+bc+ca \\
    rst&=abc-\frac{1}{3}.
\end{align*}
Now, how do we use this to our advantage to find $r^3+s^3+t^3$? For starters, we may expand
$$(r+s+t)^3=r^3+s^3+t^3+3r^2s+3r^2t+3rs^2+3rt^2+3s^2t+3st^2+6rst.$$
We're close to finding something substantial - if we can simplify the sum of the terms in the form $m^2n$. A natural way to do this is to expand
$$(r+s+t)(rs+st+rt)=r^2s+r^2t+rs^2+rt^2+s^2t+st^2+3rst.$$

Aha! Now our expression simplifies nicely:
$$(r+s+t)^3=r^3+s^3+t^3+3(r+s+t)(rs+st+rt)-3rst,$$
whence upon arranging, we have
\begin{align*}
    r^3+s^3+t^3
    &=(\alpha+\beta+\gamma)^3-3(\alpha+\beta+\gamma)(\alpha\beta+\beta\gamma+\gamma\alpha)+3(\alpha+\beta+\gamma+\frac{1}{3})\\
    &=(\alpha+\beta+\gamma)^3-3(\alpha+\beta+\gamma)(\alpha\beta+\beta\gamma+\gamma\alpha)+3(\alpha+\beta+\gamma)+1 \\
    &=\alpha^3+\beta^3+\gamma^3+1 \\
    &=\boxed{170}
\end{align*}
\end{proof}

\begin{example}[2019 AIME I P10]
For complex numbers $z_1, z_2,\cdots z_{673}$, the polynomial
$$(x-z_1)^3(x-z_2)^3\cdots(x-z_673)^3$$
can be expressed as $x^{2019}+20x^{2018}+19x^{2017}+g(x)$ where $g(x)$ is a polynomial with complex polynomials and of degree \textit{at most} $2016$. Find the value of 
$$\sum_{1\leq j\leq k\leq 673}z_jz_k.$$
\end{example}
The sum requested of us is an elementary symmetric polynomial, so we would expect to use Vieta's theorem in some way.

Let $S, P$ denote the sum of roots one and twice at a time respectively and let $Q(x)$ denote the given polynomial. Note that the required sum is $P$. Since it's weird that the given polynomial is cubed throughout, we shall consider
$$p(x)=(x-z_1)(x-z_2)\cdots(x-z_{673})=x^{673}-Sx^{672}+Px^{671}+\cdots$$

Thus, $Q(x)=[p(x)]^3$. To find $S$ and $P$, we shall expand this to find the coefficients of $x^{2018}$ and $x^{2017}$. To do this quickly, we consider how terms in 3 products can be multiplied together to give the relevant terms.

To wit, 
\begin{enumerate}
    \item $x^{2019}$ can only be formed by multiplying all three $x^{673}$, and there is only 1 way to do this.
    \item $x^{2018}$ can be formed by multiplying two $x^{673}$ and one $Sx^{672}$, and there are $\binom{3}{1}=3$ ways to do this.
    \item $x^{2017}$ can be formed by multiplying two $x^{673}$ and one $Px^{671}$, and there are $\binom{3}{1}=3$ ways to do this. It can also be formed by multiplying one $x^{673}$ and two $Sx^{672}$, giving $3$ ways as well.
\end{enumerate}
\begin{proof}
    We now have,
\begin{align*}
    Q(x)=[p(x)]^3&=(x^{673}-Sx^{672}+Px^{671}+\cdots)(x^{673}-Sx^{672}+Px^{671}+\cdots)(x^{673}-Sx^{672}+Px^{671}+\cdots) \\
    &= x^{2019}-3Sx^{2018}+3(P+S^2)x^{2017}+g(x),
\end{align*}
and thus
$$\begin{cases} -3S=20 \\ 3(P+S^2)=19 \end{cases}\Rightarrow \begin{cases} S=-\frac{20}{3} \\ P=-\frac{343}{9}\end{cases}.$$

Hence, $\boxed{P=-\frac{343}{9}}$.
\end{proof}

At this point, it's only complete if we also derive some common algebraic identities and factorisations. 
\begin{proposition}[Classic]
    In what follows, assume the summation run over $x_1,x_2,\cdots,x_n$.
    \begin{enumerate}
        \item For the bivariate $x^2+y^2=(x+y)^2-2xy$, we also have the three-variable relation $$x^2+y^2+z^2=(x+y+z)^2-2(xy+yz+zx).$$
        In general, $$\left(\sum x_i\right)^2=\sum x_i^2-2\left(\sum_{1\leq i\leq q\leq n}x_ix_j\right).$$

        \item $x^3+y^3+z^3-3xyz=\frac{1}{2}(x+y+z)[(x-y)^2+(y-z)^2+(z-x)^2]$.
        This also implies $$x^3+y^3+z^3 \geq 3xyz$$, because
        $(x-y)^2+(y-z)^2+(z-x)^2\geq 0$.
        \item Sophie-Germain Identity: $a^4+4b^4=(a^2+2b^2+2ab)(a^2+2b^2-2ab)$.
        \item Lagrange's Identity: $(a^2+b^2)(c^2+d^2)=(ac-bd)^2+(ad+bc)^2=(ac+bd)^2+(ad-bc)^2$
        \item $(a+b)^3-(a^3+b^3)=3ab(a+b)$.
        \item $(a+b)^5-(a^5+b^5)=5ab(a+b)(a^2+ab+b^2)$.
        \item $(a+b)^7-(a^7+b^7)=7ab(a+b)(a^2+ab+b^2)^2$.
        \item My personal favourite: $\frac{(a^2+bc)(b^2+ac)}{(a+c)(b+c)}+\frac{(a^2+bc)(c^2+ab)}{(a+b)(b+c)}+\frac{(b^2+ac)(c^2+ab)}{(a+b)(a+c)}=a^2+b^2+c^2$
    \end{enumerate}
\end{proposition}
\begin{example}[2016 SMO(O) P15]
    Let $p,q$ be integers such that the roots of the polynomial $f(x)=x^3+px^2+qx-343$ are real. Find the minimum value of $|p^2-2q|$.
\end{example}
This question reeks of Vieta's theorem... With some foresight, we first derive a corollary of the AM-GM inequality for three variables. From identity 2, we know that
    $$x^3+y^3+z^3 \geq 3xyz.$$

    Now take $x=a^{\frac{2}{3}}, y=b^{\frac{2}{3}}, z=c^{\frac{2}{3}}$ so that
    $$a^2+b^2+c^2\geq 3\sqrt[3]{(abc)^2}.$$
    \textit{(whispers) This is a secret tool that will help us later!}
\begin{proof}
    
    Let $a,b,c$ be the real roots of $f$, then by Vieta's theorem, $p=-(a+b+c), q=ab+bc+ca, 343=abc$.
    Thus,
    \begin{align*}
        |p^2-2q|&=|(a+b+c)^2-2(ab+bc+ca)| \\
        &=|a^2+b^2+c^2| \\
        &\geq |3\sqrt[3]{(abc)^2}| = |3\sqrt[3]{343^2}| \\
        &=\boxed{147}
    \end{align*}
\end{proof}

For the final problem of this chapter, we shall prove identity 7.
\begin{example}[Classic]
    Factorise $(a+b)^7-(a^7+b^7)$.
\end{example}
\begin{proof}
    Certainly we need some observations to reduce this problem into a more tractable form. We note that $a=0$ or $b=0$ make the expression vanish, so $ab$ is a factor of the expression. Moreover, since the degrees on $a, b$ are odd, $a=-b$ also causes the expression to vanish. Thus, $a+b$ is also a factor.

    Thus, the expression is a polynomial that readily decomposes into the factor $ab(a+b)$ of degree $3$ and another factor of degree $4$ (or possibly more factors).

    Let $a,b,c$ be roots of the cubic with $c=-(a+b)$ such that
    $$\begin{cases}
        A&=a+b+c=0 \\
        B&=ab+bc+ca=-(a^2+b^2+ab) \\
        C&=abc=-ab(a+b)
    \end{cases},$$
    whence the cubic is $$x^3+Bx-C=(x-a)(x-b)(x-c).$$
    
    This gives $a^2+b^2+c^2=(a+b+c)^2-2(ab+bc+ca)=-2B$. We now "lift" the power up to $x^7$. Suppose $t$ is one of $a,b,c$. Thus, we have
    \begin{align}
        t^7&=t(C-Bt)^2=B^2t^3-2BCt^2+C^2t \\
        &=B^2(C-Bt)-2BCt^2+C^2t \\
        &=-2BCt^2+(C^2-B^3)t+B^2C \label{12-seventh}
    \end{align}
    Summing \eqref{12-seventh} over $t=a,b,c$,
    \begin{align*}
        a^7+b^7-(a+b)^7&=-2BC(-2B)+3B^2C\\
        &=7B^2C=-7ab(a+b)(a^2+ab+b^2)^2
    \end{align*}
    whence $$(a+b)^7-(a^7+b^7)=7ab(a+b)(a^2+ab+b^2)^2.$$
\end{proof}
\end{document}