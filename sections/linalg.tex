\documentclass[../main.tex]{subfiles}
\graphicspath{{\subfix{../images/}}}

\begin{document}
We start off with a classic on matrices.
\begin{example}[Classic]
Let $A$ and $B$ be square matrices of the same order. If $A+B$ is invertible, prove that $$A(A+B)^{-1}B=B(A+B)^{-1}A.$$
\end{example}
To me, the most jarring part about this equation is the $(A+B)^{-1}$ term in the product of matrices on both sides. In this case, the problem suggests that this form of matrix multiplication is \textbf{commutative}.

Let's try to see if we can get rid of the inverse terms. Beginning with the left-hand side, we would be motivated to write $B=(A+B)-A$:
\begin{align*}
    A(A+B)^{-1}B &= A(A+B)^{-1}(A+B-A) \\
    &= A(A+B)^{-1}(A+B)-A(A+B)^{-1}A \\
    &= A+A(A+B)^{-1}A \\
\end{align*}
Similarly, on the right-hand side, with $A=(A+B)-B$:
\begin{align*}
    B(A+B)^{-1}A &= B(A+B)^{-1}(A+B-B) \\
    &= B(A+B)^{-1}(A+B)-B(A+B)^{-1}B \\
    &= B-B(A+B)^{-1}B
\end{align*}
These two expressions are quite suggestive: indeed, if we assume that the result is true, then we must apparently have 
$$A-A(A+B)^{-1}A=B-B(A+B)^{-1}B \Longrightarrow A-A(A+B)^{-1}A-(B-B(A+B)^{-1}B)=0.$$
This is quite encouraging, partly due to the symmetry, but moreso because of the terms in $(A+B)$ if we factorise the expression. Learning from the approach for the example above, where we \textit{forcibly} make things equal:
\begin{align*}
    A-A(A+B)^{-1}A-(B-B(A+B)^{-1}B) &= (A-B)-(A+B)(A+B)^{-1}(A-B) \\
    &= (A-B)-(A-B) \\
    &= 0
\end{align*}
so the result is indeed true!

The more morally upright reader may realise that this is not exactly satisfying, because we implicitly assumed that the problem is true, and then we worked backwards from there. A more pleasing approach would have continued from the first paragraph:
\begin{align*}
    A(A+B)^{-1}B &= A(A+B)^{-1}(A+B-A) \\
    &= A-A(A+B)^{-1}A = A-A(A+B)^{-1}(A+B-B) \\
    &= A-A(A+B)^{-1}(A+B)+A(A+B)^{-1}B \\
    &= A-A+A(A+B)^{-1}B \\
    &= A(A+B)^{-1}B
\end{align*}
\end{document}