\documentclass[../jarvis.tex]{subfiles}

\begin{document}
This document aims to showcase some questions that highlight the need for a variety of techniques to solve. Before proceeding, we expect some proficiency in algebra, calculus (integration and limits like L'Hopital's rule), statistics (continuous random variables and statistical distributions). As a result, most workings shown in the following sections will gloss over algebraic manipulation (unless for the sake of demonstration and effect), and the reader is invited to fill in the gaps when armed with pen and paper.

Throughout this document, we use a select-few colours to make viewing more convenient. Example problems (and its following parts) are placed in forest-green crates, and the head of each example includes the source of the problem for reference and citation. Most of the worked problems are either classical examples, or originate from school problems. Problems with parts that follow are similarly placed and are labelled as a continuation of the preceding part. Along the way, we also showcase examples of writing proofs \textit{forwards}.

The post-mortem for each worked problem is placed in a green crate with a green-black font to highlight the learning points as well as the demands and key techniques of the problem.

At the end of each section, there are a few selected problems that the reader should attempt.

The document also contains a healthy dollop of tangents that elaborate on how a worked problem can be extended to develop mathematical results.
\end{document}