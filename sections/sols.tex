\documentclass[../main.tex]{subfiles}
\graphicspath{{\subfix{../images/}}}

\begin{document}
\begin{enumerate}
    \item We have that \[m=\tan{\theta}, \frac{dm}{dx}=\sec^2{\theta}\frac{d\theta}{dx}.\]
Substituting into the above equation, we get:
\begin{align*}
    (D-x)\frac{dm}{dx}=\lambda\sqrt{1+m^2} &\Longrightarrow (D-x)\sec^2{\theta}\frac{d\theta}{dx}=\lambda\sec{\theta} \\
    &\Longrightarrow \int \sec{\theta}\,d\theta = \int \frac{\lambda}{D-x}\,dx && D-x > 0\\
    &\Longrightarrow \ln\left(\sec{\theta}+\tan{\theta}\right)=-\lambda\ln{(D-x)}+C' &&m>0 \Longrightarrow 0 < \theta < \frac{\pi}{2}\\
    &\Longrightarrow \sec{\theta}+\tan{\theta}=C(D-x)^{-\lambda} \\
    &\Longrightarrow m+\sqrt{1+m^2}=C(D-x)^{-\lambda}
\end{align*}
Finally, applying the initial conditions gives
$$m+\sqrt{1+m^2}=\left(\frac{D-x}{D}\right)^{-\lambda}=\left(1-\frac{x}{D}\right)^{-\lambda}.$$
\item While we are tempted to directly apply $\frac{d}{dx}(Vt)=\sqrt{1+m^2}$, this is most likely difficult to simplify given that we now have an expression in $Vt$. No good. It may be wiser to use the result for $\frac{dm}{dx}$, so that
\begin{align*}
    m+\sqrt{1+m^2}=\left(1-\frac{x}{D}\right)^{-\lambda} &\Longrightarrow m+\frac{D-x}{\lambda}\frac{dm}{dx}=\left(1-\frac{x}{D}\right)^{-\lambda}\\
    &\Longrightarrow \frac{dm}{dx}+\frac{\lambda}{D-x}m=\frac{\lambda}{D-x}\left(1-\frac{x}{D}\right)^{-\lambda}=\lambda D^{\lambda}(D-x)^{-\lambda-1}
\end{align*}
Now we proceed by the integrating factor approach, where the integrating factor is simply $(D-x)^{-\lambda}$:
Thus,
\begin{align*}
    m(D-x)^{-\lambda}&=\int \lambda D^{\lambda}(D-x)^{-2\lambda-1} \,dx \\
    &= \frac{D^{\lambda}(D-x)^{-2\lambda}}{2}+C \\
    m &= \frac{D^{\lambda}(D-x)^{-\lambda}}{2}+C(D-x)^{\lambda}
\end{align*}
Applying the initial conditions gives $$m=\frac{dy}{dx}=\frac{D^{\lambda}(D-x)^{-\lambda}}{2}-\frac{(D-x)^{\lambda}}{2D^{\lambda}}.$$

Finally, back-integrating gives $$y=\frac{D^{\lambda}(D-x)^{-\lambda+1}}{2(\lambda-1)}+\frac{(D-x)^{\lambda+1}}{2D^{\lambda}(\lambda+1)}+E.$$
Again, applying the initial conditions gives 
$$y=\frac{D^{\lambda}(D-x)^{-\lambda+1}}{2(\lambda-1)}+\frac{(D-x)^{\lambda+1}}{2D^{\lambda}(\lambda+1)}+\frac{D}{2(1-\lambda)}-\frac{D}{2(\lambda+1)}.$$

Indeed, we should be satisfied with this answer, because for $x \to D$, $y$ tends to the vertical line $y=\frac{D}{2(1-\lambda)}-\frac{D}{2(\lambda+1)}$. Moreover, $\frac{D}{2(1-\lambda)}-\frac{D}{2(\lambda+1)}$ is positive. If you have anxiety, a quick sketch using a graphing calculator should convince you:
\begin{figure}[H]
    \centering
    \includegraphics[scale=0.25]{graph-pursuit-sol.PNG}
    \caption{The pursuit curve for $\lambda=0.63, D=7$ (don't ask why I chose these values).}
    \label{fig:graph-pursuit-sol}
\end{figure}
\end{enumerate}

\paragraph{Solution 5.12}
The first inequality follows from the claim that $\cos{x} \leq e^{-\frac{x^2}{2}} \Longleftrightarrow f(x)=\ln{\cos{x}} \leq -\frac{x^2}{2}$ for $|x|<\frac{\pi}{2}$. This is true because $f(x)$ is:
\begin{enumerate}
    \item An even function with $f(0)=0$, and
    \item concave since $f''(x)=-\tan^2{x} < 0$.
\end{enumerate}
The second inequality follows by symmetry(!), and the last equality follows from the normal distribution $N(\mu, \sigma^2)$:
$$\frac{1}{\sigma\sqrt{2\pi}}\int_{-\infty}^{\infty}e^{-\frac{(x-\mu)^2}{2\sigma^2}} \,dx = 1$$ for the choice $\mu = 0$, $ \sigma=\frac{1}{\sqrt{2n}}$.
Hmm... cube roots again. It might be tempting to immediately cube the entire equation, but it shouldn't take more than half a minute to see that it will turn out gory. Maybe we should find another way to get rid of the cube roots.

\begin{proof}
    Let $a=\sqrt[3]{x-110}$, $b=\sqrt[3]{x-381}$, with $a\geq b$ so that
    $$\begin{cases}
        a-b=1 \\
        a^3-b^3=271
    \end{cases}
    \Longrightarrow 
    \begin{cases}
        a^2+ab+b^2=(a-b)^2+3ab=1+3ab=271 \Longrightarrow ab=90 \\
        a^2+ab+b^2=(a+b)^2-ab=(a+b)^2-90=271 \Longrightarrow a+b=\pm 19
    \end{cases}$$
Now, it's clear that we should get 2 different values of $a,b$. We shall do so using a quick trick. Suppose for now that $a+b=19$ and $ab=90$, then $a, b$ are roots of the quadratic $$x^2-19x+90=(x-9)(x-10)=0$$.

Since $a\geq b$, $a=10$ gives $x=1110$ (the reader should check that $b=9$ also gives this value of $x$). Similarly, if $a+b=-19$, then $a=-9$ gives $x=-619$.

To this end, we have exhausted all the possible cases, and so there are no more solutions. Moreover, it is easy to verify (if you know your cubes!) that both solutions satisfy the original equation.

Thus, our solutions are \boxed{\text{$x=1110$ and $x=-619$}}.
\end{proof}

\end{document}